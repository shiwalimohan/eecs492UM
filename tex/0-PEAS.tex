% Created 2012-05-25 Fri 12:21
\documentclass[11pt]{article}
\usepackage[utf8]{inputenc}
\usepackage[T1]{fontenc}
\usepackage{fixltx2e}
\usepackage{graphicx}
\usepackage{longtable}
\usepackage{float}
\usepackage{wrapfig}
\usepackage{soul}
\usepackage{textcomp}
\usepackage{marvosym}
\usepackage{wasysym}
\usepackage{latexsym}
\usepackage{amssymb}
\tolerance=1000
\providecommand{\alert}[1]{\textbf{#1}}

\title{Dicussion Outline (January 13)}
\author{Shiwali Mohan}
\date{2012-01-10 Tue}
\hypersetup{
  pdfkeywords={},
  pdfsubject={},
  pdfcreator={Emacs Org-mode version 7.8.09}}

\begin{document}

\maketitle



\section{Revision of PEAS terminology}
\label{sec-1}

performance, environment, actuators, sensors
stress on why it is important to characterize a task.
\section{Properties of task environment}
\label{sec-2}
\subsection{Observability}
\label{sec-2-1}

\begin{itemize}
\item partially observable only if the unperceived information is
  \emph{relevant} to the action choice. eg: a cleaning agent vacuuming a
  room does not need to know about the weather to plan, make a decision.
\item arises because of noisy sensors, lack of sensory data
\end{itemize}
\subsection{Single agent v/s multi agent}
\label{sec-2-2}

\begin{itemize}
\item difference between an \emph{agent} and an \emph{object}.
\item competitive, cooperative
\end{itemize}
\subsection{Determinism}
\label{sec-2-3}

\begin{itemize}
\item stochastic process generates next state
\item also related to observability of the environment
\end{itemize}
\subsection{Episodic v/s sequential}
\label{sec-2-4}

different for different formulations
\begin{itemize}
\item A single firefighting task is sequential, however one expedition
    does not affect others.
\end{itemize}
\subsection{Static v/s dynamic}
\label{sec-2-5}

environment changes during deliberation. strategy games v/s action games
\subsection{Discrete v/s continuous}
\label{sec-2-6}

continuous state, time, percepts, actions 
\subsection{Known v/s unknown}
\label{sec-2-7}

model of the environment v/s learning
\section{Examples}
\label{sec-3}


\begin{center}
\begin{tabular}{l}
 Domain                           \\
\hline
 Missionaries and Cannibals       \\
 Super Mario                      \\
 Robo-soccer                      \\
 Driving domain                   \\
 Fire-fighting domain             \\
 Hand-written Number Recognition  \\
 Siri                             \\
\end{tabular}
\end{center}
\section{Evaluation}
\label{sec-4}

\begin{itemize}
\item How to evaluate an agent's performance? Discuss wrt examples in the
  previous section. Evaluation should be done in multiple scenarios. 
  Introduce the concept of false positives/negatives; eg: hand-written
  number recognition.
\end{itemize}

\end{document}
