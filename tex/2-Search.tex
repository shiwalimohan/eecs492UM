% Created 2012-05-25 Fri 12:20
\documentclass[compress, 9pt]{beamer}
\usepackage[utf8]{inputenc}
\usepackage[T1]{fontenc}
\usepackage{fixltx2e}
\usepackage{graphicx}
\usepackage{longtable}
\usepackage{float}
\usepackage{wrapfig}
\usepackage{soul}
\usepackage{textcomp}
\usepackage{marvosym}
\usepackage{wasysym}
\usepackage{latexsym}
\usepackage{amssymb}
\tolerance=1000
\usetheme{default}
\usecolortheme[RGB={0,38,93}]{structure}
\usefonttheme{serif}
\useinnertheme{circles}
\useoutertheme[]{shadow}
\setbeamertemplate{navigation symbols}{}
\usepackage{natbib}
\usepackage{fleqn}
\usepackage{epsf}
\usepackage[dvips]{color}
\usepackage{bibentry}
\institute{Computer Science and Engineering \\ University of Michigan}
\providecommand{\alert}[1]{\textbf{#1}}

\title{Search}
\author{Shiwali Mohan}
\date{January 27, 2012}
\hypersetup{
  pdfkeywords={},
  pdfsubject={},
  pdfcreator={Emacs Org-mode version 7.8.09}}

\begin{document}

\maketitle

\begin{frame}
\frametitle{Outline}
\setcounter{tocdepth}{3}
\tableofcontents
\end{frame}


\title[Search \hspace{1em}\insertframenumber/
\inserttotalframenumber]{Full Title}



\begin{frame}
\frametitle{A Water Jug Problem}
\label{sec-1}

The agent is given two jugs, a 4-gallon one and a 3-gallon
one. Neither has any measuring markers on it. The is an infinite
source/sink of water. How can the agent get exactly 1 gallons of water
into the 3-gallon jug?
\end{frame}
\begin{frame}
\frametitle{Problem Formulation}
\label{sec-2}

\begin{itemize}
\item How would you represent state? How many states can be represented?
\item What is the initial state?
\item What is the goal state? How many states satisfy the goal test?
\item What are the actions?
\item What is the path cost?
\end{itemize}
\end{frame}
\begin{frame}
\frametitle{Search Space for Water Jug}
\label{sec-3}

\centering
\includegraphics[width=1.0\textwidth]{../images/water-jug.eps}
\end{frame}

\end{document}
