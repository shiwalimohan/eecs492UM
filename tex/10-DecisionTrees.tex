% Created 2012-05-25 Fri 12:21
\documentclass[compress, 9pt]{beamer}
\usepackage[utf8]{inputenc}
\usepackage[T1]{fontenc}
\usepackage{fixltx2e}
\usepackage{graphicx}
\usepackage{longtable}
\usepackage{float}
\usepackage{wrapfig}
\usepackage{soul}
\usepackage{textcomp}
\usepackage{marvosym}
\usepackage{wasysym}
\usepackage{latexsym}
\usepackage{amssymb}
\tolerance=1000
\usetheme{default}
\usecolortheme[RGB={0,38,93}]{structure}
\usefonttheme{serif}
\useinnertheme{circles}
\useoutertheme[]{shadow}
\setbeamertemplate{navigation symbols}{}
\usepackage{natbib}
\usepackage{fleqn}
\usepackage{epsf}
\usepackage[dvips]{color}
\usepackage{bibentry}
\institute{Computer Science and Engineering \\ University of Michigan}
\providecommand{\alert}[1]{\textbf{#1}}

\title{Decision Trees and Ensemble Learning}
\author{Shiwali Mohan}
\date{\today}
\hypersetup{
  pdfkeywords={},
  pdfsubject={},
  pdfcreator={Emacs Org-mode version 7.8.09}}

\begin{document}

\maketitle

\begin{frame}
\frametitle{Outline}
\setcounter{tocdepth}{3}
\tableofcontents
\end{frame}


\title[Search \hspace{1em}\insertframenumber/
\inserttotalframenumber]{Full Title}

\section{Review}
\label{sec-1}
\begin{frame}
\frametitle{Decision Trees Review}
\label{sec-1-1}
\begin{itemize}

\item <1-> What is a decision tree?
\label{sec-1-1-1}%
\begin{itemize}

\item <2-> a function that takes as input a vector of attribute values and returns a output value
\label{sec-1-1-1-1}%

\item <2-> assumption: inputs are discrete and output is boolean
\label{sec-1-1-1-2}%

\item <2-> given a set of examples: value assignment to attributes in input and corresponding output value
\label{sec-1-1-1-3}%
\end{itemize} % ends low level

\item <3-> Entropy of a random variable?
\label{sec-1-1-2}%
\begin{itemize}

\item <4-> measurement of uncertainity of a random variable
\label{sec-1-1-2-1}%

\item <4-> acquisition of information corresponds to reduction in entropy
\label{sec-1-1-2-2}%

\item <4-> \[H(V) = -\sum_k P(v_k)\log_2P(v_k)\]
\label{sec-1-1-2-3}%

\item <5-> $H(fair-coin)$?
\label{sec-1-1-2-4}%

\item <6-> $H(tail-only)$?
\label{sec-1-1-2-5}%
\end{itemize} % ends low level

\item <7-> $B(q)$
\label{sec-1-1-3}%
\begin{itemize}

\item <8-> entropy of a Boolean random variable that is true with a probability $q$
\label{sec-1-1-3-1}%

\item <8-> $B(q) = -(q \log_2 q + (1-q) \log_2 (1-q))$
\label{sec-1-1-3-2}%
\end{itemize} % ends low level
\end{itemize} % ends low level
\end{frame}
\section{Decision Tree}
\label{sec-2}
\begin{frame}
\frametitle{Learning to Lunch}
\label{sec-2-1}


\begin{center}
\begin{tabular}{lllll}
\hline
 Price   &  Food Type  &  Rush?  &  Results  &  Index  \\
\hline
 High    &  Sushi      &  Hurry  &  Good     &  x1     \\
 Low     &  Pizza      &  Hurry  &  Good     &  x2     \\
 Medium  &  Hamburger  &  Hurry  &  Good     &  x3     \\
 Medium  &  Pizza      &  Relax  &  Good     &  x4     \\
 High    &  Gruel      &  Relax  &  Bad      &  x5     \\
 High    &  Hamburger  &  Relax  &  Good     &  x6     \\
 Low     &  Gruel      &  Hurry  &  Bad      &  x7     \\
 Low     &  Sushi      &  Relax  &  Bad      &  x8     \\
 High    &  Pizza      &  Hurry  &  Bad      &  x9     \\
 High    &  Hamburger  &  Hurry  &  Bad      &  x10    \\
 Medium  &  Hamburger  &  Hurry  &  Bad      &  x11    \\
 Low     &  Gruel      &  Relax  &  Bad      &  x12    \\
\hline
\end{tabular}
\end{center}
\begin{itemize}

\item <2-> What is the entropy of the result attribute of the whole example set?
\label{sec-2-1-1}%
\begin{itemize}

\item <3-> $B(\frac{5}{12})$
\label{sec-2-1-1-1}%

\item <4-> $-\frac{5}{12}\log_2\frac{5}{12} - \frac{7}{12}\log_2\frac{27}{12} = 0.9803$
\label{sec-2-1-1-2}%
\end{itemize} % ends low level
\end{itemize} % ends low level
\end{frame}
\begin{frame}
\frametitle{Learning to Lunch}
\label{sec-2-2}

\small

\begin{center}
\begin{tabular}{lllll}
\hline
 Price   &  Food Type  &  Rush?  &  Results  &  Index  \\
\hline
 High    &  Sushi      &  Hurry  &  Good     &  x1     \\
 Low     &  Pizza      &  Hurry  &  Good     &  x2     \\
 Medium  &  Hamburger  &  Hurry  &  Good     &  x3     \\
 Medium  &  Pizza      &  Relax  &  Good     &  x4     \\
 High    &  Gruel      &  Relax  &  Bad      &  x5     \\
 High    &  Hamburger  &  Relax  &  Good     &  x6     \\
 Low     &  Gruel      &  Hurry  &  Bad      &  x7     \\
 Low     &  Sushi      &  Relax  &  Bad      &  x8     \\
 High    &  Pizza      &  Hurry  &  Bad      &  x9     \\
 High    &  Hamburger  &  Hurry  &  Bad      &  x10    \\
 Medium  &  Hamburger  &  Hurry  &  Bad      &  x11    \\
 Low     &  Gruel      &  Relax  &  Bad      &  x12    \\
\hline
\end{tabular}
\end{center}
\begin{itemize}

\item What is first attribute to split on?
\label{sec-2-2-1}%
\begin{itemize}

\item <2-> That attribute that gives us the most information gain, reduces the entropy by the largest amount.
\label{sec-2-2-1-1}%

\item <3-> \[Remainder(A)= \sum_{k=1}^{d}\frac{p_k+n_k}{p+n}B(\frac{p_k}{p_k+n_k})\]
\label{sec-2-2-1-2}%

\item <4-> \[Gain(A) = B(\frac{p_k}{p_k+n_k}) - Remainder(A)\]
\label{sec-2-2-1-3}%
\end{itemize} % ends low level
\end{itemize} % ends low level
\end{frame}
\begin{frame}
\frametitle{Learning to Lunch}
\label{sec-2-3}

\small

\begin{center}
\begin{tabular}{lllll}
\hline
 Price   &  Food Type  &  Rush?  &  Results  &  Index  \\
\hline
 High    &  Sushi      &  Hurry  &  Good     &  x1     \\
 Low     &  Pizza      &  Hurry  &  Good     &  x2     \\
 Medium  &  Hamburger  &  Hurry  &  Good     &  x3     \\
 Medium  &  Pizza      &  Relax  &  Good     &  x4     \\
 High    &  Gruel      &  Relax  &  Bad      &  x5     \\
 High    &  Hamburger  &  Relax  &  Good     &  x6     \\
 Low     &  Gruel      &  Hurry  &  Bad      &  x7     \\
 Low     &  Sushi      &  Relax  &  Bad      &  x8     \\
 High    &  Pizza      &  Hurry  &  Bad      &  x9     \\
 High    &  Hamburger  &  Hurry  &  Bad      &  x10    \\
 Medium  &  Hamburger  &  Hurry  &  Bad      &  x11    \\
 Low     &  Gruel      &  Relax  &  Bad      &  x12    \\
\hline
\end{tabular}
\end{center}
\begin{itemize}

\item What is first attribute to split on?
\label{sec-2-3-1}%
\begin{itemize}

\item <2-> Remainder(Type)?
\label{sec-2-3-1-1}%
\begin{itemize}

\item <3-> $\frac{2}{12}B(\frac{1}{2}) + \frac{3}{12}B(\frac{2}{3})+\frac{4}{12}B(\frac{1}{2})+\frac{3}{12}B(0) = 0.6371$
\label{sec-2-3-1-1-1}%
\end{itemize} % ends low level

\item <4-> Remainder(Price)?
\label{sec-2-3-1-2}%
\begin{itemize}

\item <5->  $\frac{5}{12}B(\frac{2}{5}) + \frac{3}{12}B(\frac{2}{3})+\frac{4}{12}B(\frac{1}{4}) = 0.90448$
\label{sec-2-3-1-2-1}%
\end{itemize} % ends low level

\item <5-> Remainder(Rush)?
\label{sec-2-3-1-3}%
\begin{itemize}

\item <6-> $\frac{6}{12}B(\frac{1}{2}) + \frac{6}{12}B(\frac{2}{3}) = 0.9591$
\label{sec-2-3-1-3-1}%
\end{itemize} % ends low level

\item <7-> Gain(Type)? Gain(Price)? Gain(Rush)?
\label{sec-2-3-1-4}%
\begin{itemize}

\item <8-> 0.3432, 0.07585, 0.0212
\label{sec-2-3-1-4-1}%
\end{itemize} % ends low level
\end{itemize} % ends low level
\end{itemize} % ends low level
\end{frame}
\section{Ensemble Learning}
\label{sec-3}
\begin{frame}
\frametitle{Ensemble of Stumps}
\label{sec-3-1}

\small

\begin{center}
\begin{tabular}{lllll}
\hline
 Price   &  Food Type  &  Rush?  &  Results  &  Index  \\
\hline
 High    &  Sushi      &  Hurry  &  Good     &  x1     \\
 Low     &  Pizza      &  Hurry  &  Good     &  x2     \\
 Medium  &  Hamburger  &  Hurry  &  Good     &  x3     \\
 Medium  &  Pizza      &  Relax  &  Good     &  x4     \\
 High    &  Gruel      &  Relax  &  Bad      &  x5     \\
 High    &  Hamburger  &  Relax  &  Good     &  x6     \\
 Low     &  Gruel      &  Hurry  &  Bad      &  x7     \\
 Low     &  Sushi      &  Relax  &  Bad      &  x8     \\
 High    &  Pizza      &  Hurry  &  Bad      &  x9     \\
 High    &  Hamburger  &  Hurry  &  Bad      &  x10    \\
 Medium  &  Hamburger  &  Hurry  &  Bad      &  x11    \\
 Low     &  Gruel      &  Relax  &  Bad      &  x12    \\
\hline
\end{tabular}
\end{center}
\begin{itemize}

\item <2-> What is the weight of examples?
\label{sec-3-1-1}%
\begin{itemize}

\item <3-> $\frac{1}{12}$
\label{sec-3-1-1-1}%
\end{itemize} % ends low level

\item <4-> What is the initial decision stump (use information gain)?
\label{sec-3-1-2}%
\begin{itemize}

\item <5-> Food type
\label{sec-3-1-2-1}%
\end{itemize} % ends low level

\item <6-> What is the classification error?
\label{sec-3-1-3}%
\begin{itemize}

\item <7-> Sushi (good), Hamburger (good), Pizza (good), Gruel (bad)
\label{sec-3-1-3-1}%

\item <7-> $\frac{4}{12}=0.333$
\label{sec-3-1-3-2}%
\end{itemize} % ends low level

\item <8-> What is the weight of the hypothesis?
\label{sec-3-1-4}%
\begin{itemize}

\item <9-> $z[1 ]= \log(\frac{1-0.333}{0.333}) = 0.695$
\label{sec-3-1-4-1}%
\end{itemize} % ends low level
\end{itemize} % ends low level
\end{frame}
\begin{frame}
\frametitle{Ensemble of Stumps}
\label{sec-3-2}

\small

\begin{center}
\begin{tabular}{lllll}
\hline
 Price   &  Food Type  &  Rush?  &  Results  &  Index  \\
\hline
 High    &  Sushi      &  Hurry  &  Good     &  x1     \\
 Low     &  Pizza      &  Hurry  &  Good     &  x2     \\
 Medium  &  Hamburger  &  Hurry  &  Good     &  x3     \\
 Medium  &  Pizza      &  Relax  &  Good     &  x4     \\
 High    &  Gruel      &  Relax  &  Bad      &  x5     \\
 High    &  Hamburger  &  Relax  &  Good     &  x6     \\
 Low     &  Gruel      &  Hurry  &  Bad      &  x7     \\
 Low     &  Sushi      &  Relax  &  Bad      &  x8     \\
 High    &  Pizza      &  Hurry  &  Bad      &  x9     \\
 High    &  Hamburger  &  Hurry  &  Bad      &  x10    \\
 Medium  &  Hamburger  &  Hurry  &  Bad      &  x11    \\
 Low     &  Gruel      &  Relax  &  Bad      &  x12    \\
\hline
\end{tabular}
\end{center}
\begin{itemize}

\item How will the weight change?
\label{sec-3-2-1}%
\end{itemize} % ends low level
\end{frame}
\begin{frame}
\frametitle{Ensemble of Stumps}
\label{sec-3-3}

\small

\begin{center}
\begin{tabular}{lllllr}
\hline
 Price   &  Food Type  &  Rush?  &  Results  &  Index  &  Weight  \\
\hline
 High    &  Sushi      &  Hurry  &  Good     &  x1     &  0.0625  \\
 Low     &  Pizza      &  Hurry  &  Good     &  x2     &  0.0625  \\
 Medium  &  Hamburger  &  Hurry  &  Good     &  x3     &  0.0625  \\
 Medium  &  Pizza      &  Relax  &  Good     &  x4     &  0.0625  \\
 High    &  Gruel      &  Relax  &  Bad      &  x5     &  0.0625  \\
 High    &  Hamburger  &  Relax  &  Good     &  x6     &  0.0625  \\
 Low     &  Gruel      &  Hurry  &  Bad      &  x7     &  0.0625  \\
 Low     &  Sushi      &  Relax  &  Bad      &  x8     &   0.125  \\
 High    &  Pizza      &  Hurry  &  Bad      &  x9     &   0.125  \\
 High    &  Hamburger  &  Hurry  &  Bad      &  x10    &   0.125  \\
 Medium  &  Hamburger  &  Hurry  &  Bad      &  x11    &   0.125  \\
 Low     &  Gruel      &  Relax  &  Bad      &  x12    &  0.0625  \\
\hline
\end{tabular}
\end{center}
\begin{itemize}

\item <1-> How will the weight change?
\label{sec-3-3-1}%
\begin{itemize}

\item <2-> $w[j] \leftarrow w[j]. \frac{error}{(1-error)}$
\label{sec-3-3-1-1}%

\item <2-> NORMALIZE(w)
\label{sec-3-3-1-2}%
\end{itemize} % ends low level

\item <3-> New entropy of the sample
\label{sec-3-3-2}%
\begin{itemize}

\item <4-> $-0.3125\log_2(0.3125) - 0.6875\log_2(0.6875)$
\label{sec-3-3-2-1}%
\end{itemize} % ends low level

\item <5-> Genrate the next stump.
\label{sec-3-3-3}%
\begin{itemize}

\item <6-> \[Remainder(A)= \sum_{k=1}^{d}\frac{p_k+n_k}{p+n}B(\frac{p_k}{p_k+n_k})\]
\label{sec-3-3-3-1}%
\end{itemize} % ends low level
\end{itemize} % ends low level
\end{frame}
\section{PAC}
\label{sec-4}
\begin{frame}
\frametitle{PAC Hypothesis}
\label{sec-4-1}
\begin{itemize}

\item <2-> Size of the hypothesis space of this problem.
\label{sec-4-1-1}%
\begin{itemize}

\item <3-> $2^{3*4*2}$
\label{sec-4-1-1-1}%
\end{itemize} % ends low level

\item <4-> If a hypothesis classifies more than 3 samples wrong, it is a bad hypothesis. What is the epsilon?
\label{sec-4-1-2}%
\begin{itemize}

\item <5-> $\frac{3}{24}$
\label{sec-4-1-2-1}%
\end{itemize} % ends low level

\item <6-> After 12 examples are observed, what is the hypothesis space?
\label{sec-4-1-3}%
\begin{itemize}

\item <7-> $2^{12}$
\label{sec-4-1-3-1}%
\end{itemize} % ends low level

\item <8-> How many of these are approximately correct?
\label{sec-4-1-4}%
\begin{itemize}

\item <9-> $1 + 12 + {12\choose2}$
\label{sec-4-1-4-1}%
\end{itemize} % ends low level

\item <10-> What is the probability that a randomly-chosen hypothesis that is consistent with the 8 examples seen so far will be approximately correct?
\label{sec-4-1-5}%
\begin{itemize}

\item <11-> $\frac{1 + 12 + {12\choose2}} {2^{12}}$\\
\label{sec-4-1-5-1}%
\end{itemize} % ends low level
\end{itemize} % ends low level
\end{frame}

\end{document}
