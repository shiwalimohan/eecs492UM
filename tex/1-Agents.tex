% Created 2012-05-25 Fri 12:21
\documentclass[11pt]{article}
\usepackage[utf8]{inputenc}
\usepackage[T1]{fontenc}
\usepackage{fixltx2e}
\usepackage{graphicx}
\usepackage{longtable}
\usepackage{float}
\usepackage{wrapfig}
\usepackage{soul}
\usepackage{textcomp}
\usepackage{marvosym}
\usepackage{wasysym}
\usepackage{latexsym}
\usepackage{amssymb}
\tolerance=1000
\providecommand{\alert}[1]{\textbf{#1}}

\title{Two banks are on the x axis. Left-bank is at lbx=0 and right-bank is}
\author{Shiwali Mohan}
\date{\today}
\hypersetup{
  pdfkeywords={},
  pdfsubject={},
  pdfcreator={Emacs Org-mode version 7.8.09}}

\begin{document}

\maketitle

\setcounter{tocdepth}{3}
\tableofcontents
\vspace*{1cm}
at rbx=20. Agent's location on the number line is at 0<=a<=20. The agent
has 5\$. If it goes to left-bank, it can get 20\$. If it goes to the
right-bank, it can get 10\$. However, in order to get to the bank the
agent has to spend money at the rate of 2*(distance from the
bank). The agent has three actions - go-left, go-right and stay. The
agent wants to maximize the money it has. Assume that the location of
the agent is an integer.

\section{Table based agent: for every location, compute the total money with}
\label{sec-1}

 the agent on going-left, going-right, stay. Choose action
 accordingly. 20 entries in the table.
\section{Threshold agent: if the location of the agent is less than 10,}
\label{sec-2}

 go-left, if greater than 15, go-right. between 10-15, stay.
\section{Change cost of moving: if the cost of moving changes from}
\label{sec-3}

 2*(distance from the bank) to 2.5*(distance from the bank). How does
 the threshold change? stay if between 8-16.
\section{Left-bank might be closed with a probability of 0.25, right-bank}
\label{sec-4}

 might be closed with a probability of 0.20. The cost of going to
 each bank is 2*(distance from the bank). The expected value of
 going to each bank changes (left-bank: 15, right-bank: 8), changing
 the `stay' threshold to 7.5-16.

\end{document}
