% Created 2012-05-25 Fri 12:20
\documentclass[compress, 9pt]{beamer}
\usepackage[utf8]{inputenc}
\usepackage[T1]{fontenc}
\usepackage{fixltx2e}
\usepackage{graphicx}
\usepackage{longtable}
\usepackage{float}
\usepackage{wrapfig}
\usepackage{soul}
\usepackage{textcomp}
\usepackage{marvosym}
\usepackage{wasysym}
\usepackage{latexsym}
\usepackage{amssymb}
\tolerance=1000
\usetheme{default}
\usecolortheme[RGB={0,38,93}]{structure}
\usefonttheme{serif}
\useinnertheme{circles}
\useoutertheme[]{shadow}
\setbeamertemplate{navigation symbols}{}
\usepackage{natbib}
\usepackage{fleqn}
\usepackage{epsf}
\usepackage[dvips]{color}
\usepackage{bibentry}
\institute{Computer Science and Engineering \\ University of Michigan}
\providecommand{\alert}[1]{\textbf{#1}}

\title{CSP and Local Search}
\author{Shiwali Mohan}
\date{February 03, 2012}
\hypersetup{
  pdfkeywords={},
  pdfsubject={},
  pdfcreator={Emacs Org-mode version 7.8.09}}

\begin{document}

\maketitle

\begin{frame}
\frametitle{Outline}
\setcounter{tocdepth}{3}
\tableofcontents
\end{frame}


\title[Search \hspace{1em}\insertframenumber/
\inserttotalframenumber]{Full Title}


\section{Contraint Satisfaction}
\label{sec-1}
\begin{frame}
\frametitle{Cryptogram}
\label{sec-1-1}

A cryptogram is a type of puzzle which consists of a short piece of
encrypted text such that each letter of the alphabet (A-Z) is mapped
to a different letter (A-Z). Spaces are preserved. The assignments are
consistent. \\
\\
\begin{itemize}

\item <1-> Cipher\\
\label{sec-1-1-1}%
\texttt{GEBO TEV E CWAACK CEGN}


\item <2-> Letter Map\\\\
\label{sec-1-1-2}%
\texttt{abcdefghijklmnopqrstuvwxyz}\\
\texttt{trl.a.m...e..by...oh.di...}


\item <3-> Decrypted string:\\
\label{sec-1-1-3}%
\texttt{MARY HAD A LITTLE LAMB}


\item <4-> Used extensively in telegrams in World War I\\
\label{sec-1-1-4}%
Zimmermann Telegram: proposal from the German Empire to Mexico to make
war againts US. intercepted by the British and decrypted

\end{itemize} % ends low level
\end{frame}
\begin{frame}
\frametitle{Cryptogram as a CSP}
\label{sec-1-2}
\begin{itemize}

\item <1-> Cryptograms can be modeled as CSPs
\label{sec-1-2-1}%

\item <2-> Example Problem\\
\label{sec-1-2-2}%
\texttt{ABC DC EFG HGIIJK FBE}


\begin{enumerate}
\item What are the variables?
\item What is the domain of those variables?
\item What are the constraints?
\end{enumerate}
\end{itemize} % ends low level
\end{frame}
\begin{frame}
\frametitle{Cryptogram as a CSP}
\label{sec-1-3}

Cipher: \texttt{ABC DC EFG HGIIJK FBE}\\
Plain text: \texttt{?}

\begin{itemize}
\item <1-> Treating each word as a variable, identify word pairs that have a constraint between them. Draw a constraint graph.
\item <2->  We use a dictionary to find the domain of each variable. \\
\texttt{Dom(ABC) = Dom(EFG) = Dom(FBE) = \{DOG, MAN, THE, HAT \}}\\
  \texttt{Dom(DC) = \{IN, MY, TO\}}\\
  \texttt{Dom(HGIIJK) = \{BATTER, SUDDEN, YELLOW\}}\\
\item <3-> Solve the cryptogram by using backtracking search (without forward checking) assigning variables from left to right.
\item <4->  How would the search change if forward checking was allowed?
\end{itemize}
\end{frame}
\section{Local Search}
\label{sec-2}
\begin{frame}
\frametitle{Traveling Salesman Problem}
\label{sec-2-1}
\begin{itemize}

\item A problem in graph theory requiring the most efficient (i.e., least total distance) Hamiltonian cycle a salesman can take through each of  cities.
\label{sec-2-1-1}%

\item <2-> No general method of solution is known, and the problem is NP-hard.
\label{sec-2-1-2}%
\end{itemize} % ends low level
\end{frame}
\begin{frame}
\frametitle{Small Traveling Salesman Problem}
\label{sec-2-2}

A traveler needs to visit 4 cities. He knows the distance between each
pair of cities. His aim to figure out the shortest route that visits
all the cities.  \\
Distances between cities. \\
\texttt{A-B 6}\\
\texttt{A-C 1}\\
\texttt{A-D 5}\\
\texttt{B-C 3}\\
\texttt{B-D 2}\\
\texttt{C-D 4}\\
\end{frame}
\begin{frame}
\frametitle{Local Search for Traveling Salesman Problem}
\label{sec-2-3}

\begin{itemize}
\item State space: a set of all possible tours, e.g. ABCD, ACBD etc
\item Operators: change position of adjacent cities within the current tour
\item Heuristic function - length of the tour
\item Initial State - ABCD
\item Distances between cities. \\
\texttt{A-B 6}\\
   \texttt{A-C 1}\\
   \texttt{A-D 5}\\
   \texttt{B-C 3}\\
   \texttt{B-D 2}\\
   \texttt{C-D 4}\\
\item Perform steepest descent.
\end{itemize}
\end{frame}
\section{Homework}
\label{sec-3}
\begin{frame}
\frametitle{Pay attention to}
\label{sec-3-1}

\begin{itemize}
\item algorithms specified in the homework questions.
\item difference between a `state' of the world and a `node'
\end{itemize}
\end{frame}

\end{document}
