% Created 2012-05-25 Fri 12:20
\documentclass[compress, 9pt]{beamer}
\usepackage[utf8]{inputenc}
\usepackage[T1]{fontenc}
\usepackage{fixltx2e}
\usepackage{graphicx}
\usepackage{longtable}
\usepackage{float}
\usepackage{wrapfig}
\usepackage{soul}
\usepackage{textcomp}
\usepackage{marvosym}
\usepackage{wasysym}
\usepackage{latexsym}
\usepackage{amssymb}
\tolerance=1000
\usetheme{default}
\usecolortheme[RGB={0,38,93}]{structure}
\usefonttheme{serif}
\useinnertheme{circles}
\useoutertheme[]{shadow}
\setbeamertemplate{navigation symbols}{}
\usepackage{natbib}
\usepackage{fleqn}
\usepackage{epsf}
\usepackage[dvips]{color}
\usepackage{bibentry}
\institute{Computer Science and Engineering \\ University of Michigan}
\providecommand{\alert}[1]{\textbf{#1}}

\title{Probability and Inference}
\author{Shiwali Mohan}
\date{\today}
\hypersetup{
  pdfkeywords={},
  pdfsubject={},
  pdfcreator={Emacs Org-mode version 7.8.09}}

\begin{document}

\maketitle

\begin{frame}
\frametitle{Outline}
\setcounter{tocdepth}{3}
\tableofcontents
\end{frame}


\title[Search \hspace{1em}\insertframenumber/
\inserttotalframenumber]{Full Title}


\section{Probability review}
\label{sec-1}
\begin{frame}
\frametitle{Terminology}
\label{sec-1-1}
\begin{itemize}

\item <1-> Prior probability
\label{sec-1-1-1}%
\begin{itemize}

\item <2-> belief in propositions in the absence of any other information
\label{sec-1-1-1-1}%

\item <2-> marginal, unconditional
\label{sec-1-1-1-2}%
\begin{itemize}

\item <3->  P(double): ?
\label{sec-1-1-1-2-1}%

\end{itemize} % ends low level
\end{itemize} % ends low level

\item <4-> Posterior Probability
\label{sec-1-1-2}%
\begin{itemize}

\item <5-> belief in propositions after evidence
\label{sec-1-1-2-1}%

\item <5-> conditional
\label{sec-1-1-2-2}%
\begin{itemize}

\item <6-> P(double/5):?
\label{sec-1-1-2-2-1}%
\begin{itemize}
\item Posterior probability
  belief in propositions after evidence
  P(double/5):?
  P(11/6):?
\end{itemize}
\end{itemize} % ends low level
\end{itemize} % ends low level
\end{itemize} % ends low level
\end{frame}
\section{Probability and Inference from Full Joint Distribution}
\label{sec-2}
\begin{frame}
\frametitle{Calculating Probabilities}
\label{sec-2-1}

Dr. Lycan Thropes, an expert in werewolves, asserts that a person is a werewolf 
iff they have hairy palms and howl at the moon. Further, he states that 25\% of the 
population either has hairy palms or howls or both, and that 20\% of the
population has hairy palms.
\begin{itemize}

\item <1-> Random variables?
\label{sec-2-1-1}%
\begin{itemize}

\item <2-> hairy palms (palms), howl (howls), werewolf (werewolf)
\label{sec-2-1-1-1}%
\end{itemize} % ends low level

\item <3-> What is the bound on probability that a randomly selected person from the population howls at the moon?
\label{sec-2-1-2}%
\begin{itemize}

\item <4-> 0.05 <= P(howls) <= 0.25
\label{sec-2-1-2-1}%
\end{itemize} % ends low level

\item <5-> If one in hundred people is a werewolf, what is the P(howls)?
\label{sec-2-1-3}%
\begin{itemize}

\item <6-> 0.06
\label{sec-2-1-3-1}%

\end{itemize} % ends low level
\end{itemize} % ends low level
\end{frame}
\begin{frame}
\frametitle{Inference on Full Joint}
\label{sec-2-2}

Noted werewolf specialist Prof. Loupgarou disagrees with Thropes, and 
provides a joint probability table for combinations of werewolves, 
hairy palms, and howling. 
\begin{itemize}

\item <2-> What is a full joint distribution?
\label{sec-2-2-1}%

\item <3-> How many entries exist in the full joint distribution?
\label{sec-2-2-2}%

\item <4-> Given the following full joint distribution. Sum to 1?\\
\label{sec-2-2-3}%
\begin{center}
\begin{tabular}{lllr}
 hairy palms  &  howls  &  werewolf  &  Probability  \\
\hline
 false        &  false  &  false     &          0.7  \\
 false        &  false  &  true      &        0.005  \\
 false        &  true   &  false     &         0.05  \\
 false        &  true   &  true      &         0.01  \\
 True         &  false  &  false     &          0.2  \\
 True         &  false  &  true      &         0.01  \\
 True         &  true   &  false     &        0.005  \\
 True         &  true   &  true      &         0.02  \\
\end{tabular}
\end{center}



\item <5-> What is the marginal probability of werewolf?
\label{sec-2-2-4}%
\begin{itemize}

\item <6-> 0.045
\label{sec-2-2-4-1}%
\end{itemize} % ends low level

\item <7-> If an individual has hairy palms and howls, what is the probability that the individual is a werewolf?
\label{sec-2-2-5}%
\begin{itemize}

\item <8-> 0.8
\label{sec-2-2-5-1}%
\end{itemize} % ends low level

\item <8-> What is the probability that a werewolf has both hairy palms and howls?
\label{sec-2-2-6}%
\begin{itemize}

\item <9-> 0.44
\label{sec-2-2-6-1}%

\end{itemize} % ends low level
\end{itemize} % ends low level
\end{frame}
\section{Bayes Net}
\label{sec-3}
\begin{frame}[fragile]
\frametitle{Inference}
\label{sec-3-1}

Events A, B, C, D, E occur with the following probabilities. Assume
conditional independence amongst variables.

\begin{verbatim}
 P(A) = 0.3
 P(B) = 0.6
 P(C|A) = 0.8
 P(C|-A) = 0.4
 P(D|A,B) = 0.7
 P(D|A,-B) = 0.8
 P(D|-A,B) = 0.1
 P(D|-A,-B) = 0.2
 P(E|C) = 0.7
 P(E|C) = 0.2
\end{verbatim}
\end{frame}
\section{Bayes Net}
\label{sec-4}

For given probabilities

P(A) = 0.3
P(B) = 0.6
P(C|A) = 0.8
P(C|-A) = 0.4
P(D|A,B) = 0.7 
P(D|A,-B) = 0.8 
P(D|-A,B) = 0.1 
P(D|-A,B) = 0.2 
P(E|C) = 0.7 
P(E|C) = 0.2

\begin{itemize}
\item Construct a Bayes Net
\end{itemize}


\begin{itemize}
\item P(D)?
\end{itemize}

P(D,A,B) + P(D,A,-B) + P(D,-A,B) + P(D,-A,-B) =
P(D|A,B) P(A,B) + P(D|A,-B) P(A,-B) + 
P(D|-A,B) P(-A,B) + P(D|-A,-B) P(-A,-B) = 
(since A and B are independent absolutely)
P(D|A,B) P(A) P(B) + P(D|A,-B) P(A) P(-B) + 
P(D|-A,B) P(-A) P(B) + P(D|-A,B) P(-A) P(-B) =
0.7*0.3*0.6 + 0.8*0.3*0.4 + 0.1*0.7*0.6 + 0.2*0.7*0.4 = 0.32


\begin{itemize}
\item P(A|C)?
\end{itemize}


P(A|C) = P(C|A)P(A) / P(C). 
Now P(C) = P(C,A) + P(C,-A) = 
P(C|A)P(A) + P(C|-A)P(-A) = 
0.8*0.3+ 0.4*0.7 = 0.52
So P(C|A)P(A) / P(C) = 0.8*0.3/0.52= 0.46.



\begin{itemize}
\item P(C|-A,E)?
\end{itemize}

P(C|-A,E) = 
P(E|C,-A) * P(C|-A) / P(E|-A) = 
P(E|C) * P(C|-A) / P(E|-A). 
Now P(E|-A) = P(E,C|-A) + P(E,-C|-A) =
P(E|C,-A) P(C|-A) + P(E|-C,-A) P(-C|-A) = (since E is independent of A given C)
P(E|C) * P(C|-A) + P(E|-C) * P(-C|-A).
So we have 
P(C|-A, E) = 
P(E|C) * P(C|-A) / (P(E|C) * P(C|-A) + P(E|-C) * P(-C|-A)) =
0.7*0.4 / (0.7 * 0.4 + 0.2 * 0.6) = 0.7
\begin{verbatim}
 P(D|-A,-B) = 0.2
 P(E|C) = 0.7
 P(E|-C) = 0.2
\end{verbatim}
\begin{frame}[fragile]
\frametitle{Inference}
\label{sec-4-1}

For given probabilities
\begin{verbatim}
 P(A) = 0.3
 P(B) = 0.6
 P(C|A) = 0.8
 P(C|-A) = 0.4
 P(D|A,B) = 0.7
 P(D|A,-B) = 0.8
 P(D|-A,B) = 0.1
 P(D|-A,-B) = 0.2
 P(E|C) = 0.7
 P(E|-C) = 0.2
\end{verbatim}
\begin{itemize}

\item P(A|C)?
\label{sec-4-1-1}%

\item <2->
\label{sec-4-1-2}%
\begin{verbatim}
 P(A|C) = P(C|A)P(A) / P(C)
 P(C) = P(C,A) + P(C,-A) 
      = P(C|A)P(A) + P(C|-A)P(-A) 
      = 0.8*0.3+ 0.4*0.7 = 0.52
 P(C|A)P(A) / P(C) = 0.8*0.3/0.52= 0.46.
\end{verbatim}

\end{itemize} % ends low level
\end{frame}
\begin{frame}[fragile]
\frametitle{Inference}
\label{sec-4-2}

For given probabilities
\begin{verbatim}
 P(A) = 0.3
 P(B) = 0.6
 P(C|A) = 0.8
 P(C|-A) = 0.4
 P(D|A,B) = 0.7
 P(D|A,-B) = 0.8
 P(D|-A,B) = 0.1
 P(D|-A,-B) = 0.2
 P(E|C) = 0.7
 P(E|-C) = 0.2
\end{verbatim}
\begin{itemize}

\item P(C|-A,E)?
\label{sec-4-2-1}%

\item <2->
\label{sec-4-2-2}%
\begin{verbatim}
 P(C|-A,E) 
 = P(E|C,-A) * P(C|-A) / P(E|-A) 
 = P(E|C) * P(C|-A) / P(E|-A).
 P(E|-A) = P(E,C|-A) + P(E,-C|-A) 
         = P(E|C,-A) P(C|-A) + P(E|-C,-A) P(-C|-A) 
           (since E is independent of A given C)
          = P(E|C) * P(C|-A) + P(E|-C) * P(-C|-A)
 P(C|-A, E) 
 = P(E|C) * P(C|-A) / (P(E|C) * P(C|-A) + P(E|-C) * P(-C|-A)) 
 0.7*0.4 / (0.7 * 0.4 + 0.2 * 0.6) = 0.7
\end{verbatim}
\end{itemize} % ends low level
\end{frame}

\end{document}
