% Created 2012-05-25 Fri 12:20
\documentclass[compress, 9pt]{beamer}
\usepackage[utf8]{inputenc}
\usepackage[T1]{fontenc}
\usepackage{fixltx2e}
\usepackage{graphicx}
\usepackage{longtable}
\usepackage{float}
\usepackage{wrapfig}
\usepackage{soul}
\usepackage{textcomp}
\usepackage{marvosym}
\usepackage{wasysym}
\usepackage{latexsym}
\usepackage{amssymb}
\tolerance=1000
\usetheme{default}
\usecolortheme[RGB={0,38,93}]{structure}
\usefonttheme{serif}
\useinnertheme{circles}
\useoutertheme[]{shadow}
\setbeamertemplate{navigation symbols}{}
\usepackage{natbib}
\usepackage{fleqn}
\usepackage{epsf}
\usepackage[dvips]{color}
\usepackage{bibentry}
\institute{Computer Science and Engineering \\ University of Michigan}
\providecommand{\alert}[1]{\textbf{#1}}

\title{Propositional Logic}
\author{Shiwali Mohan}
\date{\today}
\hypersetup{
  pdfkeywords={},
  pdfsubject={},
  pdfcreator={Emacs Org-mode version 7.8.09}}

\begin{document}

\maketitle

\begin{frame}
\frametitle{Outline}
\setcounter{tocdepth}{3}
\tableofcontents
\end{frame}


\title[Search \hspace{1em}\insertframenumber/
\inserttotalframenumber]{Full Title}


\section{Translation}
\label{sec-1}
\begin{frame}
\frametitle{Translate to Propositional Logic}
\label{sec-1-1}

\texttt{p}: You get an A on the final exam \\
\texttt{q}: You do every exercise in the book \\
\texttt{r}: You get an A in this class \\
\begin{itemize}
\item <2-> You get an A in this class, but you do not do every exercise in the
  book
\item <3-> $r \wedge -q$
\item <4-> If you got an A in this class, you must have gotten an A on the
   final.
\item <5-> $r \Rightarrow p$
\item <6-> Getting an A on the final and doing every exercise in the book is
   sufficient for getting an A in this class
\item <7-> $p \wedge q \Rightarrow r$
\item <8-> You cannot get an A in this class if you do not do every exercise
  in the book, unless you get an A on the final.
\item <9-> $-q \Rightarrow (-p \Rightarrow -r)$ or $-q \wedge -p \Rightarrow -r$
\item <10-> why?
\end{itemize}
\end{frame}
\section{Valid, Satisfiable, Unsatisfiable}
\label{sec-2}
\begin{frame}
\frametitle{Determine if valid/satisfiable/unsatisfiable}
\label{sec-2-1}

\begin{itemize}
\item <1-> RockStar $\vee$ -RockStar
\item <2-> valid, satisfiable
\item <3-> BigHouse $\wedge$ -BigHouse
\item <4-> unsatisfiable
\item <5-> Raining $\wedge$ Rainbow
\item <6-> satisfiable
\item <7-> $(-P \vee -Q) \Rightarrow ((-P \wedge Q) \vee (P \wedge -Q) \vee
  (-P \wedge -Q))$
\item <8-> valid, satisfiable
\item <9-> $((-P \vee P) \Rightarrow Q) \wedge (Q \Rightarrow (P \wedge
  -P))$
\item <10-> unsatisfiable
\end{itemize}
\end{frame}
\section{Resolution}
\label{sec-3}
\begin{frame}
\frametitle{Resolution by Contradiction}
\label{sec-3-1}

Sentence:  
\begin{itemize}
\item Heads I win; tails, you lose
\item <2-> Coin must be head or tail but not both.
\item <2-> Winning and losing are complements for both you and me.
\item <2-> Exactly one of us will win.
\end{itemize}

Prove: 
\begin{itemize}
\item I win
\end{itemize}
\end{frame}
\begin{frame}
\frametitle{Axioms}
\label{sec-3-2}

\begin{itemize}
\item Coin must be head or tail but not both.
\begin{itemize}
\item <2-> $H \Leftrightarrow -T$
\end{itemize}
\item Winning and losing are complements for both you and me.
\begin{itemize}
\item <2-> $IW \Leftrightarrow -IL$
\item <2-> $YW \Leftrightarrow -YL$
\end{itemize}
\item Exactly one of us will win.
\begin{itemize}
\item <2-> $IW \Leftrightarrow -YW$
\end{itemize}
\item Heads I win
\begin{itemize}
\item <2-> $H \Rightarrow IW$
\end{itemize}
\item Tails you lose
\begin{itemize}
\item <2-> $T \Rightarrow YL$
\end{itemize}
\item <3-> Next Step?
\end{itemize}
\end{frame}
\begin{frame}
\frametitle{KB in Conjunctive Normal Form}
\label{sec-3-3}


Knowledge Base\\

\begin{center}
\begin{tabular}{ll}
 1. $H \Leftrightarrow -T$    &  1a. $-H \vee -T$    \\
                              &  1b. $T \vee H$      \\
 2. $IW \Leftrightarrow -IL$  &  2a. $-IW \vee -IL$  \\
                              &  2b. $IL \vee IW$    \\
 3. $YW \Leftrightarrow -YL$  &  3a. $-YW \vee -YL$  \\
                              &  3b. $YL \vee YW$    \\
 4. $IW \Leftrightarrow -YW$  &  4a. $-IW \vee -YW$  \\
                              &  4b. $YW \vee IW$    \\
 5. $H \Rightarrow IW$        &  5. $-H \vee IW$     \\
 6. $T \Rightarrow LY$        &  6. $-T \vee YL$     \\
\end{tabular}
\end{center}



To prove($\alpha$): $IW$ \\
Add -$\alpha$ to KB
\end{frame}
\begin{frame}
\frametitle{Resolution}
\label{sec-3-4}


\begin{center}
\begin{tabular}{ll}
 1a. $-H \vee -T$    &  7. $T \vee IW$ [1b,5]        \\
 1b. $T \vee H$      &  8. $H \vee YL$ [1b, 6]       \\
 2a. $-IW \vee -IL$  &  9. $H \vee -YW$ [3a, 8]      \\
 2b. $IL \vee IW$    &  10. $H \vee IW$ [4b, 9]      \\
 3a. $-YW \vee -YL$  &  11. $-T \vee IW$ [1a, 10]    \\
 3b. $YL \vee YW$    &  12. $IW$ [7,11;11,13;10,18]  \\
 4a. $-IW \vee -YW$  &  13. $T$ [0, 7; 1b,15]        \\
 4b. $YW \vee IW$    &  14. $YW$ [0,4b]              \\
 5. $-H \vee IW$     &  15. $-H$ [0,5]               \\
 6. $-T \vee YL$     &  16. $-IW$ [4a,14]            \\
                     &  17. $YL$ [15,8]              \\
                     &  18. $-YW$ [15,9]             \\
                     &  17. $empty$ [12, 0; 12,16]   \\
\end{tabular}
\end{center}
\end{frame}

\end{document}
